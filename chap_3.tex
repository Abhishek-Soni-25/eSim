\chapter{Installing eSim}
\thispagestyle{empty}
\label{chap3}

\section {eSim installation in Ubuntu OS}
\begin{enumerate}
\item Download eSim installer for Linux from {\tt http://esim.fossee.in/downloads} to a local directory and unpack it. You can also unpack the  installer through the terminal. Open the terminal and navigate to the directory where this INSTALL file is located. Use the following command to unpack:
\\
\quad {\tt \$ unzip eSim-2.1.zip}
\item To install eSim and other dependencies run the following command:
\\
 \quad {\tt \$ cd eSim-2.1 \newline \$ chmod +x install-eSim.sh \newline \$ ./install-eSim.sh --install}
\item To run eSim from the terminal, type:  
\\
\quad {\tt \$ esim}
\\
  or you can double click on {\tt eSim} icon created on the Desktop after installation.
\end{enumerate}


\section {eSim installation in Windows OS}
\begin{enumerate}
\item Download \textbf{eSim-2.1\_install.exe} from  {\tt https://esim.fossee.in/downloads}
\item Disable the antivirus (if any). Now, double click on the exe file to start the installation process. If a window appears, click {\tt Yes} to complete the installation.
\item By default eSim will be installed in C drive, under an auto-generated FOSSEE Folder. Note that installation directory can neither be in "Program Files" nor contain spaces in its path.
\item \textbf{eSim} icon will be created on desktop. You can double click on the {\tt eSim} icon created on the Desktop after installation.
\end{enumerate}

